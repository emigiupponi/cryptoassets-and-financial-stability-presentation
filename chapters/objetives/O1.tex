\begin{frame}{}

    \textcolor{blue}{\textbf{Objetivo}}.\\
    \vspace{5mm}
    \begin{itemize}
        \setlength\itemsep{1em}
        
        \item[] Describir las \textcolor{blue}{características de oferta} del sistema bitcoin.\\ 
        
        \item[] Analizar a partir de un \textcolor{blue}{modelo} si la tecnología de red distribuida puede funcionar como una tecnología equivalente al dinero.\\
        \item[] Evaluar empíricamente el modelo a partir de los \textcolor{blue}{datos de la cadena de bloques} de bitcoin.
    \end{itemize}
 
    \vspace{5mm}
    \textcolor{dgreen}{\textbf{Conjetura}}. 
    \vspace{5mm}
    \begin{itemize}
    \item[] La tecnología de red distribuida DLT que subyace al sistema bitcoin puede funcionar como una \textcolor{dgreen}{tecnología funcionalmente equivalente al dinero}.
    \end{itemize}


\note{
\begin{itemize}
    \item Para esta primera parte de la investigación, se plantean tres objetivos específicos
    \item (Leer cada objetivo)
    \item A su vez, se plantea la siguiente conjetura:
    \item (Leer la conjetura)
\end{itemize}
}
    
\end{frame}
%----------