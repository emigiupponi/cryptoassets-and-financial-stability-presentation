\begin{frame}

\vspace{5mm}
\begin{displayquote}
El \textcolor{blue}{dinero} es equivalente a una forma primitiva de \textcolor{dgreen}{memoria}. 
\end{displayquote}
\raggedleft \cite{Kocherlakota1998} \\ \citetitle{Kocherlakota1998}

\vspace{5mm}
\begin{block}<2>{DLT como memoria}
La \textcolor{red}{DLT} es un registro público equivalente a un proceso de \textcolor{dgreen}{memoria} social. El dinero es equivalente a una forma primitiva de memoria social. Por lo tanto, una DLT sería un proceso tecnológico equivalente al \textcolor{blue}{dinero}.
\end{block}

\note{
\begin{itemize}
    \item Para esta primera parte de la investigación se propone retomar algunas \textbf{definiciones clásicas de dinero} como, por ejemplo, la idea de dinero como proceso de memoria social (a diferencia de la definición funcional clásica de dinero como medio de pago, reserva de valor o unidad de cuenta).
    \item En este sentido, \textbf{Kocherlakota} indica en "Money is Memory" que (leer cita)
    \item A partir de esa definición, se propone el siguiente concepto asociado a la DLT: (leer el recurado sobre DLT)
    \item En este marco, Shin (jefe de investigaciones en el BIS) indica que "la idea de dinero como tecnología de memoria aplicada en un registro universal era una propuesta teórica no observable". Pero que "los avances asociados a las DLT han hecho posible la existencia concreta de tal registro". Por lo tanto: "los activos digitales pueden ser considerados una tecnología equivalente al dinero"
    \href{http://www.overleaf.com}{Distributed ledgers and  the governance of money}
\end{itemize}
}
    
\end{frame}
%----------

\begin{frame}

\vspace{5mm}
\begin{displayquote}
El \textcolor{blue}{dinero} en la economía moderna es una forma especial de pagaré o, en términos económicos, un activo financiero. El dinero es un tipo especial de pagaré en el que todos tienen \textcolor{dgreen}{confianza}. 
\end{displayquote}
\raggedleft \cite{Mcleay2014c} \\ \citetitle{Mcleay2014c}

\vspace{5mm}
\begin{block}<2>{POW-confianza}
La DLT de bitcoin basa la \textcolor{dgreen}{confianza} en su sistema de registro en un mecanismo de consenso específico denominado \textcolor{red}{POW}.  \parencite{Ali2014}.
\end{block}

\note{
\begin{itemize}
    \item Una segunda definición de dinero relevante para esta primera parte de la investigación es la que asocia dinero a un proceso de confianza. 
    \item En este sentido, el Banco de Inglaterra, en una publicación muy influyente sobre la naturaleza del dinero de 2014 indica que (leer cita).
    \item A partir de este concepto, se propone un segundo concepto asociad a la DLT: (leer recuadro)
\end{itemize}
}
    
\end{frame}
%----------

