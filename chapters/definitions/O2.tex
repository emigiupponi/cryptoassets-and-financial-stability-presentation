\begin{frame}

\begin{displayquote}
\small
Las \textcolor{blue}{creencias} son importantes; si los jóvenes de un determinado período renuncian a algo de sus bienes por dinero, es porque piensan que la siguiente generación también lo hará.
\end{displayquote}
\raggedleft {\cite{Garratt2018},\\ \citetitle{Garratt2018}}

\vspace{5mm}
\begin{block}<2>{Creencias}
La demanda de un activo como activo monetario depende de las \textcolor{blue}{expectativas} en torno a que dicho activo pueda cumplir en el futuro con las \textcolor{blue}{funciones de medio de pago, reserva de valor y unidad de cuenta}. 
\end{block}

\note{
\begin{itemize}
    \item Para esta segunda parte de la investigación se propone retomar algunos conceptos asociados al valor del dinero fiduciario.
    \item En este sentido, \textbf{Garrat y Wallace} indican que (leer cita)
    \item A partir de esa definición, se propone el siguiente concepto de creencias (leer el recurado sobre DLT)
    \item En este sentido, por un lado, bitcoin se asemeja a las formas de dinero fiduciario actuales (como se indicó en la primera parte de la investigación). 
    \item Y, por otro, bitcoin se diferencia de los activos monetarios con valor intrínseco, como el oro, cuya demanda se encuentra vinculada, además, a su utilidad como bien de uso. 
\end{itemize}
}
    
\end{frame}
%----------

\begin{frame}

\begin{displayquote}
\small
Nada puede haber sido tan favorable a la génesis de un medio de cambio como la aceptación, por parte de \textcolor{blue}{los sujetos económicos más perspicaces y capaces}, para su propio beneficio económico, y durante un período de tiempo considerable, de bienes eminentemente vendibles con preferencia a todos los demás. 
\end{displayquote}
\raggedleft \href{https://www.jstor.org/stable/2956146}{\cite{Menger1892},\\ \citetitle{Menger1892},\\Sección VI}

\vspace{5mm}
\begin{block}<2>{Infuenciadores}
Los \textcolor{dgreen}{influenciadores} son \textcolor{blue}{actores} que \textcolor{blue}{generan información} nueva. Son actores que cuentan con \textcolor{blue}{más información} y \textcolor{blue}{mejor capacidad de análisis} que el resto de los demandantes del mercado. Son quienes determinan la tendencia de precios en el largo plazo. 
\end{block}

\note{
\begin{itemize}
    \item Una segunda idea relevante para esta segunda parte de la investigación es la que distingue a ciertos grupos por sobre el resto al momento de determinar que activos funcionarán como medios de intercambio. 
    \item En este sentido, por ejemplo, Menger, en una publicación muy influyente sobre el origen del dinero indica que (leer cita).
    \item A partir de estas ideas de Menger, se propone definir a los actores influyentes como (leer recuadro)
\end{itemize}
}
    
\end{frame}
%----------

\begin{frame}

\begin{displayquote}
La sabiduría mundana enseña que es mejor para la \textcolor{blue}{reputación} \textcolor{blue}{fracasar convencionalmente} que tener éxito de forma no convencional.
\end{displayquote}
\raggedleft \cite{Keynes1936TheExpectation} \\ \citetitle{Keynes1936TheExpectation}\\ Sección V

\vspace{5mm}
\begin{block}<2>{Seguidores}
Los \textcolor{dgreen}{seguidores} son los agentes que \textcolor{blue}{imitan} el comportamiento de los actores influyentes. El comportamiento de los seguidores se encuentra asociado al comportamiento en el precio de corto plazo. 
\end{block}

\note{
\begin{itemize}
    \item Finalmente, otra idea relevante para esta segunda parte de la investigación se asocia al comportamiento de actores imitadores.  
    \item En este sentido, por ejemplo, Keynes, en la última parte de la sección 5 del capítulo 12 de Teoría General (sección donde se desarrollan las ideas del concurso de belleza) indica que (leer cita).
    \item A partir de estas ideas de Keynes, se propone definir a los seguidores como (leer recuadro)
\end{itemize}
}
    
\end{frame}
%----------