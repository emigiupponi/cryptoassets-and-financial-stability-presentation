\begin{frame}

\vspace{5mm}
\begin{displayquote}
\textcolor{blue}{Sustitución de monedas} se define como el uso en un país determinado de múltiples monedas como \textcolor{blue}{medio de cambio}. 
\end{displayquote}

\vspace{5mm}
\begin{displayquote}
El término \textcolor{dgreen}{dolarización} también se utiliza con frecuencia para indicar que una moneda extranjera sirve como unidad de cuenta o como \textcolor{dgreen}{depósito de valor}, y no necesariamente como medio de cambio.
\end{displayquote}
\vspace{5mm}
\raggedleft \cite{Calvo1992}\\ \citetitle{Calvo1992}

\note{
\begin{itemize}
    \item Para esta tercera parte de la investigación se propone retomar algunos conceptos asociados a la sustitución de activos monetarios.
    \item En este sentido, \textbf{Calvo y Vegh} proponen dos definiciones complementarias (leer citas)
    \item Esta distinción entre sustitución de activos como medios de pago o como reserva de valor será relevante para la presente parte de la investigación.  
\end{itemize}
}
    
\end{frame}
%----------