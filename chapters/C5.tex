\subsection{Conclusiones generales}
%------------------------

\begin{frame}{Conclusiones}

\begin{itemize}
    \item[] Es posible que \textcolor{blue}{bitcoin} puede funcionar como una \textcolor{blue}{tecnología equivalente al dinero} debido a sus características de oferta que lo asimilan a un \textcolor{blue}{registro público de transacciones}. 
    \vspace{5mm}
    \item[] Es posible que la \textcolor{dgreen}{demanda de bitcoin} se encuentre asociada a las opiniones de un grupo de \textcolor{dgreen}{actores influyentes} porque, al igual que el dinero fiduciario, la demanda de bitcoin se asocia a la \textcolor{dgreen}{creencia} que pueda funcionar como medio de cambio y reserva de valor.
    \vspace{5mm}
    \item[] Un incremento en la  \textcolor{orange}{volatilidad en la demanda} de bitcoin puede generar incrementos en la \textcolor{orange}{volatilidad del tipo de cambio} entre pesos y dólares en Argentina (ev. conceptual).
    \end{itemize}
    
\note{
\begin{itemize}
    \item El tipo de cambio se aprecia o deprecia dependiendo de los cambios en la demanda relativa de bitoin frente al peso (por los alcanzados por las restricciones) o frente al dolar (por los no alcanzados por las restricciones). De otra manera, la demanda de bitcoin responde a factores externos (bitcoin por dolar) e internos (bitcoin por pesos)
    \item Bitcoin intenta situarse como moneda de reserva global por sus características de diseño y influencia de actores. En Argentina, bitcoin se podría situar como un activo de reserva alternativo al dolar lo que podría generar incrementos en la volatilidad del tipo de cambio. 
\end{itemize}
}
    
\end{frame}
%----------

\subsection{Trabajos futuros}
%----------------------------

\begin{frame}{}

En futuros trabajos se espera:
    \vspace{5mm}
\begin{itemize}
    \item[] Estimar la demanda de bitcoin en Argentina a partir de \textcolor{blue}{información de operadores locales y globales} que permita separar la demanda a cambio de pesos y la demanda a cambio de dólares.
        \vspace{5mm}
    \item[] Avanzar en modelos de economía abierta que incorporen \textcolor{dgreen}{incertidumbre}.
        \vspace{5mm}
    \item[] Articular los modelos propuestos con análisis desde la \textcolor{orange}{teoría de juegos}.  
    \end{itemize}
    
\note{
\begin{itemize}
    \item 
\end{itemize}
}
    
\end{frame}