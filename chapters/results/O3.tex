\begin{frame}{}

    \vspace{5mm}
    \begin{itemize}
        \setlength\itemsep{1em}
        \item[] La \textcolor{blue}{\textbf{conclusión principal}} a partir de la evaluación conceptual es que incrementos en la \textcolor{blue}{volatilidad de la demanda de bitcoin} pueden generar incrementos en la \textcolor{blue}{volatilidad en el tipo de cambio} ARS/USD. 
        \item[] De manera complementaria, la \textcolor{dgreen}{\textbf{evaluación conceptual}} indica que un incremento de la demanda de ciudadanos que \textcolor{dgreen}{demandaban dólares (pesos)} generará una \textcolor{dgreen}{apreciación (depreciación)} del tipo de cambio.

    \end{itemize}

\note{
\begin{itemize}
    \item Luego de realizar la evaluación conceptual y empírica en esta tercera parte de la investigación, se  llega a las siguientes conclusiones: (leer filmina)
    \item La conclusión general. Este resultado tiene como condición necesaria un incremento en el nivel de adopción generalizado de bitcoin en Argentina. 
    \item Evaluación empírica: hay que mejorar los datos
    \item No se observa volatilidad en la demanda a partir de google trends
    \item Se enceuntra asociación entre incrementos en las búsquedas de google bitocin y apreciación en el tipo de cambio (muy inicial). 
\end{itemize}
}
    
\end{frame}
%----------