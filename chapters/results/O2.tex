\begin{frame}{}

    \vspace{5mm}
    \begin{itemize}
        \setlength\itemsep{1em}
        \item[] La \textcolor{blue}{\textbf{conclusión principal}} es que la \textcolor{blue}{demanda de bitcoin se encuentra asociada} del comportamiento de un grupo de actores \textcolor{blue}{influyentes} y un grupo de \textcolor{blue}{seguidores}.
        \item[] La \textcolor{dgreen}{\textbf{evaluación conceptual}} corrobora la conjetura a partir de la introducción en el modelo OLG la ecuación del \textcolor{dgreen}{parámetro de influencia ($\lambda$)} en función del comportamiento de un grupo de actores influyentes y del comportamiento de un grupo de seguidores. 
        \item[] La \textcolor{orange}{\textbf{evaluación empírica}} corrobora la conjetura a partir de la \textcolor{orange}{correlación} entre el \textcolor{orange}{precio} de bitcoin y los \textcolor{orange}{tweets} de los actores influyentes ponderados por la estructura de red asociada al comportamiento de los seguidores.   

    \end{itemize}

\note{
\begin{itemize}
    \item Luego de realizar la evaluación conceptual y empírica en esta segunda parte de la investigación, se  llega a las siguientes conclusiones: (leer filmina)
\end{itemize}
}
    
\end{frame}
%----------